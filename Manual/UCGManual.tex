\documentclass{article}
\usepackage{amsmath}
\usepackage{amsfonts}
\usepackage{hyperref}

\title{Manual for the Ultra-Coarse-Graining at Rapid Local Equilibrium (UCG-RLE) Theory and LAMMPS Codes}
\author{Jaehyeok Jin (with James F. Dama)\\ jhjin@uchicago.edu}
\date{Drafted: Feb 1, 2021 \\ Updated: Sep 26, 2024}

\begin{document}

\maketitle

\section{Introduction}
This is the manual for the Ultra-Coarse-Graining at Rapid Local Equilibrium (UCG-RLE) theory and its implementation in LAMMPS. The manual describes the theoretical background, the technical implementation, and the practical aspects of using this code in molecular simulations.

\section{Algorithm}
The UCG force components introduce technical difficulties that have been solved by a new UCG algorithm based on Ref. [1]. At the rapid local equilibrium limit, the effective UCG potential \( U \) is written as:
\[
U = \sum_{I} p^I U(R_I)
\]
where \( p^I \) is a set of per-particle (substate) probabilities occupying a given state \( s_I \). These probabilities are evaluated based on the local environment of the coarse-grained (CG) particles using a pair neighbor list, as implemented in LAMMPS.

The derivatives of the UCG potential contribute to the force components:
\[
\nabla U = \sum_{I} \nabla_I \left( p^I U(R_I) \right)
\]
In contrast to conventional CG force fields, \( p^I \) depends on the CG configurations, complicating the force component evaluation.

\subsection{Many-body interactions}
The many-body nature of the UCG force field arises because the substate probability of CG site \( I \) is determined by the local density of CG site \( I \), as a function of \( R_J \):
\[
p^I(R_J) = f\left( \rho(R_J) \right), \quad \rho(R_J) = \sum_{J \in N(I)} \omega(R_{IJ})
\]
where \( N(I) \) denotes the neighbors of CG site \( I \). The force on particle \( I \) can be expressed as:
\[
F_I = - \nabla_I U = - \sum_{J \in N(I)} \sum_{s_I, s_J} p^I p^J \nabla_I U(s_I, s_J, R_{IJ})
\]
This leads to three-body interactions, requiring the Newton pair option to be turned off in LAMMPS.

\subsection{Simplification using the chain rule}
The UCG force calculation can be simplified using the chain rule:
\[
\nabla U = \sum_{I} \nabla_I \left( p^I(R_J) \right)
\]
This avoids the need for nested loops, which would otherwise slow down the computational performance.

\subsection{Practical Warnings}
\begin{enumerate}
    \item This algorithm requires turning off the Newton pair during the UCG simulation.
    \item The Virial computed from this pair style may be problematic, so we do not recommend using this pair style for constant NPT simulations.
    \item Do not use large time steps for UCG propagation, as it can lead to instability and energy drift.
\end{enumerate}

\section{Usage}
The current pair style is suitable for single-component liquids with two internal states. However, it can also model multi-component UCG systems, such as heterogeneous liquid-liquid interfaces.

For example, the \texttt{state\_setting.txt} file might contain:

\begin{verbatim}
2 4
2 density no_entropy
12.5 11.00
0.0
2 density no_entropy
12.5 7.00
0.0
\end{verbatim}

The file starts with a header specifying the number of CG types and UCG states. For each CG type, the internal states and parameters are defined.

\subsection{Input file: state\_setting.txt}
The \texttt{state\_setting.txt} file defines parameters for CG types. The header provides the total number of CG types and states:
\begin{verbatim}
nCGtot nUCGtot
nUCG(1) CVtype EntropyOption
rho_th(1)  r_th(1)
One-BodyEntropy
nUCG(n) CVtype EntropyOption
rho_th(n)  r_th(n)
One-BodyEntropy
\end{verbatim}

\section{Editing the Pair Style}
In the \texttt{pair\_table\_ucgrle.cpp} file, each function computes terms such as the counting function, the probability function, and their derivatives.

\begin{itemize}
    \item \texttt{threshold\_prob\_and\_partial\_from\_cv}: Controls the substate probability function and its derivatives.
    \item \texttt{compute\_proximity\_function}: Computes the proximity function \( \omega(R) \).
    \item \texttt{compute\_proximity\_function\_der}: Computes the derivative of the proximity function.
    \item \texttt{compute}: The main routine, which sequentially computes terms such as the state probability and forces.
\end{itemize}

\section{Relevant Literature}
Several studies have used this UCG-RLE code in different contexts:
\begin{enumerate}
    \item \textbf{Ref. 1:} Introduces the UCG-RLE theory and its efficient implementation, demonstrating toy models and the hydrophobic association of neopentane in methanol.
    \item \textbf{Ref. 5:} Develops the UCG-RLE theory for liquid-vapor and liquid-liquid interfaces, showing enhanced transferability compared to bulk models.
    \item \textbf{Ref. 6:} Implements the UCG-RLE model for hydrogen-bonding liquids, using an extended MS-CG framework.
    \item \textbf{Ref. 7:} Links the UCG-RLE model to mean-field theories, allowing for transferability across different temperatures and phase transitions.
\end{enumerate}

\section{References}
\begin{itemize}
    \item J. F. Dama, J. Jin, and G. A. Voth, \textit{The Theory of Ultra-Coarse-Graining 3: Coarse-Grained Sites with Rapid Local Equilibrium of Internal States}, J. Chem. Theory Comput. 2017, 13, 1010-1022.
    \item A. Davtyan, J. F. Dama, A. V. Sinitskiy, and G. A. Voth, \textit{The Theory of Ultra-Coarse-Graining 2: Numerical Implementation}, J. Chem. Theory Comput. 2014, 10, 5265−5275.
    \item G. S. Ayton, G. A. Voth, \textit{Hybrid Coarse-Graining Approach for Lipid Bilayers}, J. Phys. Chem. B 2009, 113, 4413−4424.
    \item A. Das, L. Lu, H. C. Andersen, G. A. Voth, \textit{The Multiscale Coarse-Graining Method X: Improved Algorithms for Constructing Coarse-Grained Potentials}, J. Chem. Phys. 2012, 136, 194115.
    \item J. Jin, G. A. Voth, \textit{Ultra-Coarse-Grained Models for Interfacial Systems}, J. Chem. Theory Comput. 2018, 14, 2180-2197.
    \item J. Jin, Y. Han, G. A. Voth, \textit{Ultra-Coarse-Grained Liquid State Models with Implicit Hydrogen Bonding}, J. Chem. Theory Comput. 2018, 14, 6159-6174.
    \item J. Jin, A. Yu, G. A. Voth, \textit{Temperature and Phase Transferable Bottom-Up Coarse-Grained Models}, J. Chem. Theory Comput. 2020, 16, 6823-6842.
\end{itemize}

\end{document}
